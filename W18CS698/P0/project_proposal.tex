\documentclass[11pt]{article}

\usepackage[margin=1in]{geometry}
\geometry{lettersize}

\usepackage{graphicx}
\usepackage{float}
\usepackage{wrapfig}

\linespread{1.2}

\renewcommand{\maketitle}{
    \begin{flushright}
        {\LARGE \textbf{Physical based sound simulation for animations}}


    {\large \textsc{Wenhan Zhu (Cosmos)} \\ \textit{University of Waterloo}}
    \\ \today

    \end{flushright}
}

\begin{document}

\maketitle

\section*{Introduction}
I'm interested in physical based sound simulations and would like to do a project on it. The goal of physical based sound simulation is to have automatically generated sound synthesize to match the animation. Such techniques allow real time sound generation and does not require sound designers to design sound for different scenarios. I would like to explore on this area and hopefully create a working demo for some simple designs.

\section*{Related Work}
Recently (SIGGRAPH 2016) there is a course focused on this topic.\cite{James2016} It covers the major research done in recent years and I think it's a great place to start learning about the field. Some other earlier research are also done by van den Doel et. al.\cite{Doel2001}

\section*{Approaches}
I find that for each of the simulations there could be multiply models. I found 2 part of them very interesting. The first is the sound generated by model vibration and the other is the sound generated from rigid body. Model vibration can be approached from using differential equations to solve wave equations. And rigid body is where things get very interesting. The paper from Brien, et. el. \cite{Brien2002} explore this topics and shows that we can generate realistically sound by arbitarily shaped objects based only on geometric description and material parameters. 







\bibliographystyle{unsrt}

\bibliography{sample}

\end{document}
